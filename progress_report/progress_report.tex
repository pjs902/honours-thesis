\documentclass[12pt,letterpaper]{article}
\usepackage{fullpage}
\usepackage[top=2cm, bottom=4.5cm, left=2.5cm, right=2.5cm]{geometry}
\usepackage{amsmath,amsthm,amsfonts,amssymb,amscd}
\usepackage{lastpage}
\usepackage{fancyhdr}
\usepackage{mathrsfs}
\usepackage{xcolor}
\usepackage{graphicx}
\usepackage{listings}
\usepackage{hyperref}
\usepackage{gensymb}
\usepackage{siunitx}
\usepackage{microtype}

\setlength{\parindent}{0.0in}
\setlength{\parskip}{0.05in}

% Edit these as appropriate
\newcommand\course{PHYS 4790 (Honours Thesis)}
\newcommand\hwnumber{1}                 % <-- Number
\newcommand\Name{Peter Smith}           % <-- Name
\newcommand\ANum{A00426605}             % <-- A#

\pagestyle{fancyplain}
\headheight 35pt
\lhead{\Name \\ \ANum}
\chead{\textbf{Progress Report}}    % Title
\rhead{\course \\ \today}    % Date
\lfoot{}
\cfoot{}
\rfoot{\small\thepage}
\headsep 1.5em


\newcommand{\software}[1]{\textrm{\MakeUppercase{#1}}}

\begin{document}

\begin{center}
	\Large{\textbf{The Effects of Binary Stars on Inferred Remnant Populations in Globular
			Clusters}}
\end{center}



\section{Completed Work}
\paragraph{}
Up to this point, I have completed several components of the project. I've now fully implemented a
method that allows me to generate mass functions that include binary stars. In particular, this
method allows for arbitrary binary fraction and mass ratio distributions. This means that we can
generate mass functions that include realistic binary populations tailored specifically to any
cluster of our choosing.

Additionally, I have implemented a method to determine how these binary systems would have been
observed in the stellar mass function data that we use. When the stellar mass function data is
recorded, the authors assume that all stars are single and assign the system a mass based on its
luminosity using an isochrone fit to the cluster. This means that binary systems will be counted as
a single system with a mass corresponding to the sum of the luminosities of each component. Using
MIST isochrones, I replicate this process for our synthetic binaries, calculating first the
individual luminosities of the components and then the observed masses given the sum of the
luminosities.


\section{Remaining Work}
\paragraph{}

The process of fully adapting the GCFit package to work with our modified models requires a bit more
work, in particular using the observed masses of the binary systems to adjust the density profiles
in the models. After GCFit is fully working with our models the last remaining thing is to actually
use our models and fit them to real clusters. In particular, would be most interesting to compare
models with and without binaries to each other when they are both fitted to the same data. 47\,Tuc
is an ideal candidate for this as we already have all the needed data and the binary population is
relatively well constrained. Time permitting, NGC\,3201 would be another good candidate as it has
been observed for several epochs with the MUSE spectrograph, giving us an estimate of its binary
population based on radial velocity measurement. As an additional experiment, fitting some models
with the black hole content forced to zero but with a larger binary fraction would enable us to test
some claims that the dynamical effects of binaries can mimic those of a small population of black
holes.

In addition to the remaining work that needs to be implemented, the Methods, Results and Discussion
chapter need to be written. The Methods chapter can be mostly written immediately as most of the
methods are fully implemented and working. The Results and Discussion chapters require actual
results and thus can probably be only planned out and have introductions written. The writing will
happen at the same time as the implementation as much as possible to avoid having three full
chapters left to write after the model fits are completed.



\section{Updated Timeline}
\begin{itemize}
	\item Get the realistic binary populations working. \checkmark
	      \begin{itemize}
		      \item Implement equal mass binaries. \checkmark
		      \item Implement flat $q$ distribution. \checkmark
		      \item Implement arbitrary $q$ distribution. \checkmark
		      \item Truncate $q$ distribution according to the smallest possible value of $q$. \checkmark
	      \end{itemize}
	\item Project Summary. \checkmark
	\item Literature Review. \checkmark
	\item Integrate models with binaries with GCFit
	      \begin{itemize}
		      \item Re-bin models with binaries to reduce the runtime. \checkmark
		      \item Keep track of the overall binary populations within the rebinned models. \checkmark
		      \item Use isochrones to get the apparent luminosity of the binary systems. \checkmark
		      \item Use the ``observed mass" of the binary systems in order to re-scale the
		            number density profiles for mass function likelihoods. (Feb 18th)
		      \item Update GCFit to support the modified models (Feb 18th)
	      \end{itemize}

	\item Fit models (Feb 25th). We can now fit our models with binaries to observations of real
	      clusters. In particular, re-fitting clusters that we've already analyzed with our new
	      models which incorporate binaries should give us a good idea of the effects that
	      realistic binary populations have on our models.
	\item Methods Section (Ongoing)
	\item Results Section (Pending Model Fits)
	\item Discussion Section (Pending Results)
	\item Thesis Draft (March 18th)
	\item End date (April 4th)
\end{itemize}

\end{document}