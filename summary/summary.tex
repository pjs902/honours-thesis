\documentclass[12pt,letterpaper]{article}
\usepackage{fullpage}
\usepackage[top=2cm, bottom=4.5cm, left=2.5cm, right=2.5cm]{geometry}
\usepackage{amsmath,amsthm,amsfonts,amssymb,amscd}
\usepackage{lastpage}
\usepackage{fancyhdr}
\usepackage{mathrsfs}
\usepackage{xcolor}
\usepackage{graphicx}
\usepackage{listings}
\usepackage{hyperref}
\usepackage{gensymb}
\usepackage{siunitx}

\setlength{\parindent}{0.0in}
\setlength{\parskip}{0.05in}

% Edit these as appropriate
\newcommand\course{PHYS 4790 (Honours Thesis)}
\newcommand\hwnumber{1}                 % <-- Number
\newcommand\Name{Peter Smith}           % <-- Name
\newcommand\ANum{A00426605}             % <-- A#

\pagestyle{fancyplain}
\headheight 35pt
\lhead{\Name \\ \ANum}
\chead{\textbf{Project Summary}}    % Title
\rhead{\course \\ \today}    % Date
\lfoot{}
\cfoot{}
\rfoot{\small\thepage}
\headsep 1.5em


\newcommand{\software}[1]{\textrm{\MakeUppercase{#1}}}

\begin{document}


\section{Description}
\begin{itemize}
	\item (Working) Title: The effects of binary stars on recovered remnant populations.
	\item Hypothesis/Research Question: So I think the goal is basically to see what the effects
	      of realistic binary populations are on \software{limepy} models and by extension, see
	      what the effects on the recovered remant population are.
	\item Goals and Objectives: Create realistic binary populations, make some toy models to
	      demonstrate the effects, fit models with realistic binary populations to observations,
	      compare remant populations.
\end{itemize}

\section{Motivation and Rationale}
Right now, our models don't account for binaries at all, some studies suggest that binaries could
mimic the effect of heavy remants (Do we need citations in the summary?). By including realistic
binary populations, we hope to get better estimates for the heavy remants. Discuss the (few)
observations that we have and the limits of the observations, as well as things that just aren't
well constrained at all like the distribution of mass ratios in GCs.


\section{Methodology}
\begin{itemize}
	\item Approach to problem
	\item Techniques/Methods


\end{itemize}

Basically we're just going the shift the mass around according to the binary fractions and chosen
mass ratio distribution. the models will then be fit in the usual way with GCfit but will need to
take special care with mass function fitting. Check how much detail we want here, probably want to,
at the minimum, justify why we can just shift mass around and basically treat binaires as single
stars.

\section{Timelime}
\begin{itemize}
	\item Basic reading and planning \checkmark
	\item Get the realistic binary populations working
	      \begin{itemize}
		      \item Currently have it working with binaries defined by mass fraction
		            \checkmark
		      \item Use the correct binary fraction so that we can compare to observations
		            (1-3 weeks? less straightforward than I hoped)
	      \end{itemize}
	\item Project Summary (Nov 1st)
	\item Toy models (1-2 weeks)
	\item Use mass-luminosity relations to get the apparent mass of the binary system in order
	      to fit on MF data (1-2 weeks)
	\item Modify the GCfit code to allow for the mass function to be fit to the observed mass of
	      the binary system (I'm thinking this might be the longest part, maybe a few weeks?)
	\item Fit models (1-2 weeks, should easy once we modify GCfit)
	\item Literature Review (Ongoing - Jan 31st)
	\item Progress Report (Feb 7th)
	\item Thesis Draft (Ongoing - March 18th)
	\item End date (April 4th)
\end{itemize}

\end{document}