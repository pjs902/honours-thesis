\documentclass[12pt,letterpaper]{article}
\usepackage{fullpage}
\usepackage[top=2cm, bottom=4.5cm, left=2.5cm, right=2.5cm]{geometry}
\usepackage{amsmath,amsthm,amsfonts,amssymb,amscd}
\usepackage{lastpage}
\usepackage{fancyhdr}
\usepackage{mathrsfs}
\usepackage{xcolor}
\usepackage{graphicx}
\usepackage{listings}
\usepackage{hyperref}
\usepackage{gensymb}
\usepackage{siunitx}

\setlength{\parindent}{0.0in}
\setlength{\parskip}{0.05in}

% Edit these as appropriate
\newcommand\course{PHYS 4790 (Honours Thesis)}
\newcommand\hwnumber{1}                 % <-- Number
\newcommand\Name{Peter Smith}           % <-- Name
\newcommand\ANum{A00426605}             % <-- A#

\pagestyle{fancyplain}
\headheight 35pt
\lhead{\Name \\ \ANum}
\chead{\textbf{Project Summary}}    % Title
\rhead{\course \\ \today}    % Date
\lfoot{}
\cfoot{}
\rfoot{\small\thepage}
\headsep 1.5em


\newcommand{\software}[1]{\textrm{\MakeUppercase{#1}}}

\begin{document}


\section{Description}
\begin{itemize}
	\item (Working) Title: The effects of binary stars on recovered remnant populations.
	\item Hypothesis/Research Question: So I think the goal is basically to see what the effects
	      of realistic binary populations are on \software{limepy} models and by extension, see
	      what the effects on the recovered remant population are.
	\item Goals and Objectives: Create realistic binary populations, make some toy models to
	      demonstrate the effects, fit models with realistic binary populations to observations,
	      compare remant populations.
\end{itemize}

\section{Motivation and Rationale}
\paragraph{}
Our current, state-of-the-art equilibrium models assume that all stars are single and make no effort
to include the effect of binaries. The highest mass binary systems are of comperable mass to the
heaviest white dwarfs and neutron stars and should also mass segregate at a similar rate. Having a
large centrally concentrated population of binaries could very well reduce the need for a large
central population of heavy remnants. By including the dynamical effects of binaries in our models
we hope to recover more accurate remnant populations in present day clusters.


\paragraph{}
Binary populations in globular clusters are, in general, not very well constrained. Even for
clusters for which we have large photometric surveys and multi-epoch radial velocity measurements,
we are often only able to observe binary systems with high mass ratios. This results in loose
constraints on the total binary fraction and weak constraints on the mass ratio distribution above $
q = 0.5$, with essentially no constraints on the mass ratio distribution for $q < 0.5$. There are a
few options in the literature for likely mass ratio distributions, the one with the most
observational motivation being a flat mass ratio distribution where all value of $q$ are equally
likely. Other options include random sampling from the IMF or adopting the observed mass fraction
distribution from the solar neighbourhood. By adopting a mass ratio prescription and binary fraction
we can generate a realistic population of binaries that is likely to mimic those found in present
day clusters.



\section{Methodology}
\paragraph{}
Because of the high densities and old ages of globular clusters, we can safely assume that all long
period, loosely bound binaries have been ionized. This allows us to treat binary star systems as a
single system for the purposes of our models, a binary system of two $0.5 M_\odot$ stars should
behave like a single star of $ 1 M_\odot$. This means that in order to replicate the effects of
binary stars in our models we can simply shift some of the stars in our binned mass function into
bins of higher mass according to the chosen binary population.
\paragraph{}
When we fit these modified models to observations, we need to take special care when comparing to
the available stellar mass function data. The method used to collect the mass function data is to
assume all observed stars are single stars and to assign each star a mass based on its luminosity.
This means that when we compare our models to the data we will need to convert the dynamical mass of
the binaries into an "observed mass" which is related to the total luminosity of the binary system.
These binary bins will then contribute to the single star bin which is closest to the "observed
mass" of the binary bin for the purposes of computing mass function likelihoods.

\section{Timelime}
\begin{itemize}
	\item Basic reading and planning \checkmark
	\item Get the realistic binary populations working
	      \begin{itemize}
		      \item Currently have it working with binaries defined by mass fraction
		            \checkmark
		      \item Use the correct binary fraction so that we can compare to observations
		            (1-3 weeks? less straightforward than I hoped)
	      \end{itemize}
	\item Project Summary (Nov 1st)
	\item Toy models (1-2 weeks)
	\item Use mass-luminosity relations to get the apparent mass of the binary system in order
	      to fit on MF data (1-2 weeks)
	\item Modify the GCfit code to allow for the mass function to be fit to the observed mass of
	      the binary system (I'm thinking this might be the longest part, maybe a few weeks?)
	\item Fit models (1-2 weeks, should easy once we modify GCfit)
	\item Literature Review (Ongoing - Jan 31st)
	\item Progress Report (Feb 7th)
	\item Thesis Draft (Ongoing - March 18th)
	\item End date (April 4th)
\end{itemize}

\end{document}