\documentclass[12pt,letterpaper]{article}
\usepackage{fullpage}
\usepackage[top=2cm, bottom=4.5cm, left=2.5cm, right=2.5cm]{geometry}
\usepackage{amsmath,amsthm,amsfonts,amssymb,amscd}
\usepackage{lastpage}
\usepackage{fancyhdr}
\usepackage{mathrsfs}
\usepackage{xcolor}
\usepackage{graphicx}
\usepackage{listings}
\usepackage{hyperref}
\usepackage{gensymb}
\usepackage{siunitx}
\usepackage{microtype}

\setlength{\parindent}{0.0in}
\setlength{\parskip}{0.05in}

% Edit these as appropriate
\newcommand\course{PHYS 4790 (Honours Thesis)}
\newcommand\hwnumber{1}                 % <-- Number
\newcommand\Name{Peter Smith}           % <-- Name
\newcommand\ANum{A00426605}             % <-- A#

\pagestyle{fancyplain}
\headheight 35pt
\lhead{\Name \\ \ANum}
\chead{\textbf{Project Summary}}    % Title
\rhead{\course \\ \today}    % Date
\lfoot{}
\cfoot{}
\rfoot{\small\thepage}
\headsep 1.5em


\newcommand{\software}[1]{\textrm{\MakeUppercase{#1}}}

\begin{document}

\begin{center}
	\Large{\textbf{The Effects of Binaries on Recovered Remnant Populations}}
\end{center}


\section{Motivation and Rationale}
\paragraph{}
Current, state-of-the-art equilibrium models typically assume that all stars are single and make no
effort to include the dynamical effects of binary systems. The highest mass binary systems are of
comparable mass to the heaviest white dwarfs and neutron stars and should also mass segregate at a
similar rate. Having a large centrally concentrated population of binaries could very well reduce
the need for a large central population of heavy remnants. By including the dynamical effects of
binaries in our models we hope to recover more accurate remnant populations for present-day globular
clusters.


\paragraph{}
Binary populations in globular clusters are, in general, not very well constrained. Even for
clusters for which we have large photometric surveys or multi-epoch radial velocity measurements, we
are often only able to observe binary systems with high mass ratios. This results in loose
constraints on the total binary fraction and weak constraints on the mass ratio distribution above
$q = 0.5$, with essentially no constraints on the mass ratio distribution for $q < 0.5$. There are a
few options in the literature for likely mass ratio distributions, the one with the most
observational motivation being a flat mass ratio distribution where all values of $q$ are equally
likely. Other options include random sampling from the IMF or adopting the observed mass fraction
distribution from the solar neighbourhood. By adopting a mass ratio prescription and binary fraction
we can generate a realistic population of binaries that is likely to mimic those found in
present-day clusters.



\section{Methodology}
\paragraph{}
Because of the high densities and old ages of globular clusters, we can safely assume that all long
period, loosely bound binaries have been ionized by the present day. This allows us to treat binary
star systems as a single system for the purposes of our models, a binary system of two $0.5
\mathrm{M}_\odot$ stars should behave like a single star of $ 1 \mathrm{M}_\odot$. This means that
in order to replicate the effects of binary stars in our models we can simply shift some of the
stars in our binned mass function into bins of higher mass according to the properties of the chosen
binary population.
\paragraph{}
When we fit these modified models to observations, we need to take special care when comparing them
to the available stellar mass function data. The method used to collect the mass function data is to
assume all observed stars are single stars and to assign each star a mass based on its luminosity.
This means that when we compare our models to the data we will need to convert the dynamical mass of
the binaries into an "observed mass" which is related to the total luminosity of the binary system.
These binary bins will then contribute to the single star bin which is closest to the "observed
mass" of the binary bin for the purposes of computing mass function likelihoods.

\section{Timelime}
\begin{itemize}
	\item Get the realistic binary populations working (November 5th)
	      \begin{itemize}
		      \item Implement equal mass binaries. \checkmark
		      \item Implement flat $q$ distribution. \checkmark
		      \item Truncate $q$ distribution according to smallest possible value of $q$.
		      \item Tweak current implementation to better handle the lowest mas stars.
	      \end{itemize}
	\item Project Summary (Nov 1st) \checkmark
	\item Integrate models with binaries with GCFit (January 7th).
	      \begin{itemize}
		      \item Re-bin models with binaries to reduce the runtime (November 19th)
		      \item Keep track of the overall binary populations within the rebinned models (Dec 3rd)
		      \item Use isochrones to get the apparent luminosity of the binary systems and
		            use this to estimate the "observed mass" of the binary systems in order
		            to re-scale the number density profiles for mass function likelihoods.
		            (January 7th)
	      \end{itemize}
	\item Literature Review (Jan 31st)
	\item Toy models (Nov 19). We can run a few models with realistic binary populations and
	      compare them to models without binaries. This will let us look at the effects of the
	      binary systems without needing to fit models to real data.
	\item Fit models (Jan 28th). Once we've modified the mass function likelihood to work with
	      binaries, we can fit our models with binaries to real observations of clusters and see
	      how the model parameters and remant populations change with the inclusion of realistic
	      binary populations.
	\item Progress Report (Feb 7th)
	\item Thesis Draft (March 18th)
	\item End date (April 4th)
\end{itemize}

\end{document}