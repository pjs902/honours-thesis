\newcommand{\evolvemf}{\code{evolve\_mf }}

\section{Generating mass functions}

\subsection{Single Star Mass Functions}
The bulk of project deals with generating mass functions to use as inputs to the \code{LIMEPY}
models, we do this in two main steps, we first generate a present day mass function comprised of
only single stars, and we then modify it to include binary stars.

To generate the mass functions comprised of single stars we use the \evolvemf algorithm from
\code{SSPTools}\footnote{\url{www.github.com/pjs902/ssptools}}, a publicly available package for
working with simple stellar populations. We provide to \evolvemf the initial mass function slopes,
breakpoints, cluster age, cluster metallicity and escape velocity, and it returns a sampled present
day mass function which fully accounts for stellar evolution and effects like the natal-kicks of black
holes.

\ps{This section is proving to be hard to write, ill do the other ones first}

\subsection{Binary Mass Functions}

In order to include binary stars in our mass functions we make use of the assumption that for the
vast majority of their interaction with other objects, binary systems behave essentially as point
masses given the fact that they are tightly bound. This means that in order to replicate the effects
of a binary population in our mass function, we simply need to shift some of the mass in single
stars into heavier bins which act as the "binary bins".

\section{Fitting Models to Data}

\subsection{Data}

Go over all the data we use

Grab most of this from the paper

\subsection{GCfit}

Maybe? Discuss GCfit and nested sampling/MCMC


\subsection{Likelihoods}

Go over Gaussian likelihoods

check paper for this

\subsubsection{Fitting Mass Functions to Observations}


Here just need to discuss how binaries are handled in the observational data and how we mimic that
process with our synthetic binaries.