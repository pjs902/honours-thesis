\newcommand{\evolvemf}{\code{evolve\_mf}}



\section{Data}

We use a wide range of data to constrain the parameters of our models. In general, for both 47\,Tuc
and NGC\,3201 we use archival kinematic data from ground based spectroscopy, number density profiles from
\emph{Gaia} and stellar mass function data from HST photometry.

\subsection{Kinematics and density profiles}

\subsubsection{Proper motion dispersion profiles}

For 47 Tuc, we use two sets of {\it Hubble Space Telescope} (HST) proper motion data that are not
available for NGC\,3201. To probe the inner regions of the cluster we use the proper motion
dispersion profiles (both tangential and radial components) from \citet{Watkins2015} which are based
on a catalogue of proper motions of bright stars from \citet{Bellini2014}. These dispersion profiles
are built from stars brighter than the main sequence turn-off ($0.85 \ \mathrm{M}_\odot$ and $0.8 \
	\mathrm{M}_\odot$ for 47\,Tuc and NGC\,3201). To probe the kinematics in the outer regions of the
cluster, we also use the data from \citet{Heyl2017}, for which the mean mass of the measured stars
is $0.38 \ \mathrm{M}_{\odot}$. The outer proper motion data also allows us to constrain the amount
of radial anisotropy present in the cluster, which can mimic the effect of central dark mass in
isotropic models by raising the central velocity dispersion \citep{Zocchi2017}.


\subsubsection{Line-of-sight velocity dispersion profiles}

For both 47\,Tuc and NGC\,3201 we use the line-of-sight velocity dispersion profiles from
\citet{Baumgardt2018} to further constrain the kinematics of the clusters. These dispersion profiles
are based on archival ESO/VLT and Keck spectra along with previously published radial velocity data
from the literature. As these radial velocity samples are dominated by bright stars, we assume that
the velocity dispersion profile traces the kinematics of upper main-sequence and evolved stars in
our models.

\subsubsection{Number density profiles}
We use the number density profiles from \citet{DeBoer2019} to constrain the size and structural
parameters of the clusters. These profiles are made up of a combination of cluster members based on
Gaia DR2 data in the outer regions and data from various literature sources in the central regions.
The Gaia data used only includes bright stars ($m > 0.6 \ \mathrm{M}_\odot$, for both clusters) and
the literature data is dominated by bright stars, therefore in our models we assume the profiles
probe the distribution of upper main sequence and evolved stars.

\subsection{Stellar mass functions}

As a constraint on the global present-day stellar mass function of the clusters, we use a
compilation of HST based stellar mass function data from
Baumgardt\footnote{\url{https://people.smp.uq.edu.au/HolgerBaumgardt/globular/}} (2021, priv.
comm.), which represent an updated and augmented version of the stellar mass functions found in
\citet{Sollima2017}. This compilation is made up of several HST fields at varying distances from the
cluster centre. These fields extend out to $14 '$ and $8.33 '$ from the cluster centres for 47\,Tuc
and NGC\,3201 respectively and cover a mass range of $0.16 - 0.8 \ \mathrm{M}_\odot$. The large
radial and mass ranges allow us to constrain the degree mass segregation in the cluster.


\subsection{Binary Data}

\ps{Expand this, if we even need it here, might be better to discuss this when we discuss generating
	the binaries} When creating realistic binary populations we use the data from \citet{Milone2012} to
inform our choices of binary fraction and mass ratio distribution. For NGC\,3201 we have
\citet{Giesers2019} as an additional measurement of the binary population.





\section{Generating mass functions}

The bulk of project deals with generating mass functions to use as inputs to the \code{LIMEPY}
models, we do this in two main steps, we first generate a present day mass function comprised of
only single stars, and we then modify it to include binary stars.

\subsection{Single Star Mass Functions}

To generate the mass functions comprised of single stars we use the \evolvemf{} algorithm from
\code{SSPTools}\footnote{\url{www.github.com/pjs902/ssptools}}, a publicly available package for
working with simple stellar populations.

The \evolvemf{} algorithm combines precomputed grids of stellar evolution models and isochrones to
accurate model the evolution of the initial mass function, fully including the effects of stellar
evolution, mass loss due to escaping stars and dynamical ejections. The algorithm returns a sampled
mass functions for a requested time, ideal for use in the \code{LIMEPY} models.

We provide to \evolvemf{} the initial mass function slopes and breakpoints, the cluster age,
metallicity and escape velocity, as well as parameters which control the mass loss due to escaping
stars the specific binning to be used when the present day mass function is sampled.



\subsection{Binary Mass Functions}

In order to include binary stars in our mass functions we make use of the assumption that for the
vast majority of their interaction with other objects, binary systems behave essentially as point
masses given the fact that they are tightly bound. This means that in order to replicate the effects
of a binary population in our mass function, we simply need to shift some of the mass in single
stars into heavier bins which act as the "binary bins".






\section{Fitting Models to Data}

\subsection{GCfit}

Maybe? Discuss GCfit and nested sampling/MCMC


\subsection{Likelihoods}

Go over Gaussian likelihoods

check paper for this

\subsubsection{Fitting Mass Functions to Observations}


Here just need to discuss how binaries are handled in the observational data and how we mimic that
process with our synthetic binaries.