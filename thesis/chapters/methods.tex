\newcommand{\evolvemf}{\code{evolve\_mf}}



\section{Data}


We use a wide range of data to constrain the parameters of our models. In general, we use archival
kinematic data from ground based spectroscopy, number density profiles from \emph{Gaia}, stellar
mass function data from HST photometry and pulsar timing data.

\subsection{Kinematics and density profiles}

\subsubsection{Proper motion dispersion profiles}

We use two sets of {\it Hubble Space Telescope} (HST) proper motion data. To probe the inner regions
of the cluster we use the proper motion dispersion profiles (both tangential and radial components)
from \citet{Watkins2015} which are based on a catalogue of proper motions of bright stars from
\citet{Bellini2014}. These dispersion profiles are built from stars brighter than the main sequence
turn-off ($0.85 \ \mathrm{M}_\odot$ and $0.8 \ \mathrm{M}_\odot$ for 47\,Tuc and NGC\,3201). To
probe the kinematics in the outer regions of the cluster, we also use the data from
\citet{Heyl2017}, for which the mean mass of the measured stars is $0.38 \ \mathrm{M}_{\odot}$. The
outer proper motion data also allows us to constrain the amount of radial anisotropy present in the
cluster, which can mimic the effect of central dark mass in isotropic models by raising the central
velocity dispersion \citep{Zocchi2017}.


\subsubsection{Line-of-sight velocity dispersion profiles}

We use the line-of-sight velocity dispersion profiles from \citet{Baumgardt2018} to further
constrain the kinematics of the clusters. These dispersion profiles are based on archival ESO/VLT
and Keck spectra along with previously published radial velocity data from the literature. As these
radial velocity samples are dominated by bright stars, we assume that the velocity dispersion
profile traces the kinematics of upper main-sequence and evolved stars in our models.

\subsubsection{Number density profiles}
We use the number density profile from \citet{DeBoer2019} to constrain the size and structural
parameters of the cluster. These profiles are made up of a combination of cluster members based on
Gaia DR2 data in the outer regions and data from various literature sources in the central regions.
The Gaia data used only includes bright stars ($m > 0.6 \ \mathrm{M}_\odot$, for both clusters) and
the literature data is dominated by bright stars, therefore in our models we assume the profiles
probe the distribution of upper main sequence and evolved stars.

\subsection{Stellar mass functions}

As a constraint on the global present-day stellar mass function of the cluster, we use a
compilation of HST based stellar mass function data from
Baumgardt\footnote{\url{https://people.smp.uq.edu.au/HolgerBaumgardt/globular/}} (2021, priv.
comm.), which represent an updated and augmented version of the stellar mass functions found in
\citet{Sollima2017}. This compilation is made up of several HST fields at varying distances from the
cluster centre. These fields extend out to $14 '$ and $8.33 '$ from the cluster centres for 47\,Tuc
and NGC\,3201 respectively and cover a mass range of $0.16 - 0.8 \ \mathrm{M}_\odot$. The large
radial and mass ranges allow us to constrain the degree mass segregation in the clusters.

\subsection{Pulsar Data}

For 47\,Tuc, we make use of its large population of millisecond pulsars (MSPs) to place further
constraints on its mass distribution. We use both the spin and orbital period timing solutions from
\citet{Freire2017}, \citet{Ridolfi2016} and \citet{Freire2018}. We also consider the dispersion
measures of the pulsars which, when combined with internal gas models from \citet{Abbate2018}, allow
us to constrain the line-of-sight position of the pulsars within the cluster. The work surrounding
the use of pulsar data to constrain the models was performed as part of an earlier project and so
will not be discussed in too much detail in this project.



\subsection{Binary Data}

In order to create realistic binary populations we use the data from \citet{Milone2012} to inform
our choices of binary fraction and mass ratio distribution. For 47\,Tuc this means a flat mass
ration distribution and a binary fraction of roughly $2\%$. Because this estimate of the binary
fraction is so small, we will use it as a lower limit for the binary fraction and also test a case
where the binary fraction is around $10\%$ like in NGC\,3201, where \citet{Milone2012} find a binary
fraction of around $10\%$, again with a flat mass ratio distribution.





\section{Generating mass functions}

The bulk of project deals with generating mass functions to use as inputs to the \code{LIMEPY}
models, we do this in two main steps, we first generate a present-day mass function comprised of
only single stars, and we then modify it to include binary stars.

\subsection{Single Star Mass Functions}


To generate the mass functions comprised of single stars we use the \evolvemf{} algorithm from
\code{SSPTools}\footnote{\url{www.github.com/pjs902/ssptools}} (first presented in
\citealt{Balbinot2018}), a publicly available package for working with simple stellar populations.

The \evolvemf{} algorithm combines precomputed grids of stellar evolution models and isochrones to
accurately model the evolution of a given initial mass function, fully including the effects of
stellar evolution as well as mass loss due to escaping stars and dynamical ejections. The algorithm
returns a sampled mass function at a requested evolutionary time, ideal for use in the \code{LIMEPY}
models.

We parameterize the mass function as a broken power-law with breakpoints at $0.5 \mathrm{M}_\odot$
and $1.0 \mathrm{M}_\odot$. We provide to \evolvemf{} the initial mass function slopes and
breakpoints, the cluster age, metallicity and escape velocity, as well as parameters which control
the mass loss due to escaping stars and the specific binning to be used when the present day mass
function is sampled. Figure \ref{fig:2/evolve_mf} shows the evolution of a mass function over a span
of $10 \ \mathrm{Gyr}$.

\begin{figure}
    \centering
    \includegraphics[width=\textwidth]{figures/evolve_mf.png}
    \caption{The evolution of a typical mass function from $t=0$ to $t=10000 \ \mathrm{Myr}$.
        The stellar bins are plotted in green while the remnant bins are plotted in black,
        the current main-sequence turn-off is plotted a dashed black line. As the mass
        function ages, more and more main sequence stars evolve into remnants. Lower right:
        The evolution of the total mass of the mass function is plotted as a fraction of the
        initial mass. Mass loss is dominated by the effects of stellar evolution but also
        has contribution from dynamical ejected and escaping stars.}
    \label{fig:2/evolve_mf}
\end{figure}


\subsection{Binary Mass Functions}

In order to include binary stars in our mass functions we make use of the assumption that for the
vast majority of their interactions with other objects, binary systems behave essentially as point
masses due to the fact that they are tightly bound. This means that in order to replicate the effects
of a binary population in our mass function, we simply need to shift some of the mass in single
stars into heavier bins which act as the "binary bins".


We split this process up into several steps. First we divide the total binary fraction among the
values of $q$ in the requested mass ratio distribution. We weight the $f_b$ values assigned to the
individual values of $q$ by the chosen mass ration distribution, a flat mass ratio distribution
would have the total binary fraction divided evenly among the values while a "solar distribution"
(see \citealt{Fisher2005}) would have a significantly higher portion of the total $f_b$ assigned to
equal mass binaries ($q=1$). Figure \ref{fig:2/q-dists} shows the resulting mass ratio distributions
using this method.

\begin{figure}
    \centering
    \includegraphics[width=\textwidth]{figures/q-dists.png}
    \caption{The resulting mass ratio distributions for the "flat" and "solar" mass ratio prescriptions.
        Both distributions are truncated and lowered at $q=0.2$ due to the relative lack low very low mass
        stars within the mass functions, making the creation of binary systems with a very low mass ratio
        impossible.}
    \label{fig:2/q-dists}
\end{figure}


After we have calculated the individual binary fractions for each value of $q$, we then go through
each bin of main-sequence stars and attempt to make binaries. The companion mass for a given bin is
calculated using the current value of $q$ and the number of binaries to make is calculated using the
binary fraction for the current value of $q$. After the companion mass and number of binaries are
set, we then find the closest bin to the companion mass and subtract from the primary and companion
bins the mass corresponding to the calculated number of binaries, adding the subtracted mass to a
new bin with a mean mass equal to the sum of each binary component.


We repeat this process for each bin of main-sequence stars until all bins have a binary fraction
corresponding to the weighted $f_b$ of the current value of $q$. We do this process for each value
of $q$ in the mass ratio distribution, resulting in each main-sequence bin having a binary fraction
equal to the total requested binary fraction and a mass ratio distribution identical to the
requested distribution.


This process tends to create on the order of 150 new bins in our mass function which dramatically
increases the runtime of the \code{LIMEPY} models. In order to prevent this we group together binary
bins of similar masses, forming 15 binary bins containing binary systems of similar total mass but
differing mass ratios. Figure \ref{fig:2/shifted-mf} shows the original main sequence bins, plotted
with the modified main sequence bins, binary bins and rebinned binary bins.


\begin{figure}
    \centering
    \includegraphics[width=0.8\textwidth]{figures/shifted-mf.png}
    \caption{The main-sequence portion of a mass function before and after binaries are added. The blue
        circles are the original main sequence and the crosses are the modified main sequence. The orange
        crosses show the single stars after mass has been removed to create binaries and the many green
        crosses are the binary bins that are initially created. The red crosses are the rebinned binary bins
        which are actually used in the computation of the \code{LIMEPY} models.}
    \label{fig:2/shifted-mf}
\end{figure}



\section{Fitting Models to Data}


To fit our models to the data we use the \code{GCfit}
package\footnote{\url{www.github.com/nmdickson/gcfit}}. \code{GCfit} provides a uniform interface
for fitting \evolvemf{} and \code{LIMEPY} models to observations of clusters using either MCMC or
Nested Sampling.

For this project we use the MCMC backend which is powered by \code{EMCEE}
\citet{Foreman-Mackey2013,Foreman-Mackey2019}. We use 1024 walkers, initialized at a reasonable
estimate of the best-fit parameters. We run the chain for at least 2000 steps and discard the
initial burn-in period.

\subsection{Likelihoods}

The majority of the likelihood functions we use are simple Gaussian likelihoods of the following form:

\begin{equation}
    \ln \left(\mathcal{L}\right)=\frac{1}{2}
    \sum_{r}\left(\frac{\left(\sigma_{\mathrm{obs}}(r)
        -\sigma_{\mathrm{model}}(r)\right)^{2}}{\delta \sigma_{\mathrm{obs}}^{2}(r)}
    -\ln \left(\delta \sigma_{\mathrm{obs}}^{2}(r)\right)\right)
\end{equation}

Where $\mathcal{L}$ is the likelihood, $\sigma$ is the line-of-sight velocity dispersion, $r$ is the
projected distance from the cluster centre, and $\delta \sigma$ is the uncertainty in the velocity
dispersion. The likelihoods for other observables are formulated in the same way, and the specifics
are discussed in \code{GCfit}'s documentation\footnote{\url{gcfit.readthedocs.io}}. The total
likelihood is therefore the sum of all the log-likelihoods for each set of observations.

For the mass function and number density likelihoods we include additional nuisance and scaling
terms to account for extra sources of error in the mass function data and the effects of potential
escapers at the cluster boundary.

\subsubsection{Pulsar Likelihood}

\ps{I think we might want more detail here}

As stated previously, the development of a method to use pulsar acceleration measurements to
constrain the models was performed as part of an earlier project, but we will provide a brief
description of the likelihood function.

In order to assign a likelihood to a particular acceleration measurement we first use the
\texttt{LIMEPY} models to generate a line-of-sight acceleration profile for the given model. We then
use this acceleration profile to interpolate the possible line-of-sight positions for the pulsar.
These line-of-sight positions are then assigned a likelihood based on a Gaussian centred at the
line-of-sight position as calculated from the dispersion measure (DM) of the pulsar with a width
equal to the uncertainty of the DM-based line-of-sight position. The likelihood is the sum of the
height of the Gaussian at the interpolated line-of-sight positions from the model.




\subsection{Fitting Mass Functions to Observations}

When the mass function data was originally collected, the mass was recorded based on the position of
the star on an isochrone fit to the cluster (see \citealt{Sollima2017} for details). This means that
any binary stars in the sample are recorded as single stars a mass corresponding to a star with the
average colour of the two binary components and a luminosity corresponding to the sum of the two
components.

Additionally, when we move mass around to create binary bins we remove mass from the surface density
profiles which would be compared to the mass function data. In order to compensate for these
effects, we rescale the surface density profiles to include the stars which are in binary bins,
according to how they would have been observed using the observational method described above.

In order to determine the observed mass we use a grid of MIST isochrones \citep{Dotter2016,Choi2016}
computed at a range of metallicities, at the age of the cluster. We use the isochrone closest to the
model parameters to determine the luminosity of the binary components and then use the isochrone to
determine the observed mass of the combined luminosities. We then scale the surface density profiles
of the main-sequence bin which most closely matches the observed mass of the binary system by the
total mass of the binary system which allows us to correct for both effects.