\ps{I'm thinking an intro to globular clusters, then to modelling GCs with discussion of binaries,
	then to observations of binaries in GC}

\section{Globular Clusters}

Globular clusters (GCs) are dense, spheroidal collection of stars bound by their own self-gravity.
GCs are found in most galaxies and in the Milky Way are located both in the halo and the disk. GCs
typically represent some of the oldest stellar populations in the universe and are usually in excess
of 10 billion years old.

Mention mass segregation

\subsection{Binaries in Globular Clusters}

Mention why we expect binaries in GCs to be different from field binaries. (cite a field binary and GC binary paper here)

Some dynamical effects of binaries, mention that we're focusing on hard binaries that we can treat
as point masses, not so much the long-period binaries that provide significant energy through
hardening during interactions.

\subsection{Observations of Binary Stars in Globular Clusters}

Main sequence photometry \citet{Milone2012}

Radical Velocity Searches \citet{Giesers2019}

Time-Series Photometry \citet{Albrow2001}

\section{Modelling Globular Clusters}

\subsection{Evolutionary Models}
N-body (Nobody6 ref? Maybe just a Baumgardt ref?)

Monte-Carlo, (CMC \citet{Rodriguez2021}) (MOCCA \citet{Hypki2013,Giersz2013})
\subsection{Equilibrium Models}

Jeans Models (maybe a Sollima or Watkins ref?)

DF models (LIMEPY \citet{Gieles2015})

