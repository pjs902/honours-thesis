
\section{Globular Clusters}


Globular clusters (GCs) are dense, spheroidal collection of stars bound by their own self-gravity.
GCs are found in most galaxies, with the Milky Way hosting roughly 150, mostly located in the outer
halo. GCs typically represent some of the oldest stellar populations in the universe and are usually
in excess of 10 billion years old. The dynamics of globular clusters are almost entirely governed by
the interactions between individual cluster members, with small effects from the galactic potential
of its host galaxy as well as mass loss due to stellar evolution. Despite the fact that two-body
relaxation is essentially the sole driver of the evolution of GCs, they nonetheless display a wide
range dynamical phenomena. Among these phenomena, mass segregation is a process through which
heavier objects migrate to the centre of a cluster and lighter objects move to the outer regions. As
objects interact with each other, their energies will tend to equalize which leads to heavier
objects slowing down and lighter objects speeding up \citep{Heggie2003}. This process leads to the
core of cluster containing a much higher proportion of high-mass stars and heavy remnants than the
rest of the clusters. Figure \ref{fig:1/ngc7006} show the globular cluster NGC\,7006, imaged by the
Hubble Space Telescope's Advanced Camera for Surveys. The dense core of the cluster is clearly
visible and is made up of tens of thousands of stars.

\begin{figure}
	\centering
	\includegraphics[width=0.8\textwidth]{figures/c42.jpg}
	\caption{The globular cluster NGC 7006 imaged by the Hubble Space Telescope's Advanced
		Camera for Surveys, photo courtesy of ESA/Hubble \& NASA}
	\label{fig:1/ngc7006}
\end{figure}



The study of stellar remnants in globular clusters has far-reaching implications for diverse fields
of astrophysics. Globular clusters are one of the most commonly proposed candidates to host
intermediate-mass black holes (IMBHs). Due to  the effects of mass segregation and the high
densities of the cores of globular clusters, the cores of globular cluster are an ideal environment
for mergers of compact objects. These mergers can be detected through their resultant gravitational
waves and the expected rates for gravitational wave events depend significantly on the compact
object populations in globular clusters. These mergers are also thought to be one of the most
promising formation channels for IMBHs, a so-far undetected class of black holes whose masses fall
between those of stellar mass black holes and those of supermassive black holes. The formation of
these black holes have important implications for understanding the formation of the supermassive
black holes that we find at the centre of galaxies.


This work builds on a previous project I worked on which used pulsar timing data to constrain the
properties of the globular cluster 47\,Tuc. In that work, we were able to place strong limits on the
mass in dark remnants (black holes, neutron stars, white dwarfs) within the cluster, establishing a
strong upper limit on the black hole content specifically. While this project was able to fully
account for effects like mass segregation and uncertain mass functions, one limitation of the models
that it used (which we will discuss in the following section) was the assumption that all objects
within the cluster are single. Because the masses of binary stars are higher than the typical masses
of objects within the cluster, they too will mass segregate to the core of the cluster like heavy
stellar remnants. While the binary fraction in 47\,Tuc is expected to be quite low
\citep{Milone2012}, the effects that a centrally concentrated population of binary stars might have
on the recovered remnant content of the cluster is still somewhat unclear and worth investigating.



\begin{figure}
	\centering
	\includegraphics[width=0.8\textwidth]{figures/radial_mean_mass.png}
	\caption{Mean mass of objects within a realistic model of the globular cluster 47\,Tuc, as a
		function of radius. The concentration of high-mass objects in the central regions of
		the cluster is obvious, as is the preference for low-mas objects in the outskirts of
		the cluster.}
	\label{fig:1/radial_mean_mass}
\end{figure}




\section{Modelling Globular Clusters}

\paragraph{}

\ps{TODO: go through and make sure any jargon is properly explained}

When modelling globular clusters, there are generally two approaches commonly used. The first is to
model the entire evolutionary history of the cluster from initial conditions to the present-day. The
most commonly employed versions of these "evolutionary models" are direct N-body integration (see
for example \citet{Baumgardt2017a}) which directly calculate the gravitational interactions between
each object in the cluster and Monte-Carlo models (see \citet{Rodriguez2021} or \cite{Hypki2013})
which approximate the gravitational interactions between object according to the method of
\citet{Henon1971}. While these models provide insight into the dynamical history of the cluster,
they are very computationally expensive with even the fastest models taking on the order of a day to
model a realistic globular cluster \citep{Rodriguez2021}.

The second approach is to model just the present-day conditions of the cluster. These models, which
we call "equilibrium models", capture none of the dynamical history of the cluster but fully
describe the present-day state of the cluster. These equilibrium models are much less
computationally demanding than evolutionary models. Their relative efficiency allows us to explore a
significantly larger parameter space when fitting the models to observations to constrain the
present-day properties of a cluster. In particular, it is worth highlighting that by using
equilibrium models we are able to vary the stellar mass function of the cluster as well as the black
hole and remnant retention fractions with more flexibility than what might be possible with
evolutionary models, due to the computational cost of computing extensive grids of evolutionary
models with many parameters varied in the initial conditions (e.g. various stellar initial mass
functions, initial cluster radii, masses, etc.).

The comparative efficiency of these models further enables the use of statistical fitting techniques
like MCMC or Nested Sampling which would be prohibitively expensive to use with evolutionary models.
This means that instead of a computing a grid of models and finding the "best-fitting" model we can
instead recover posterior distributions for key cluster parameters.


In this work we use the \code{LIMEPY} family of models presented by \citet{Gieles2015}. The
\code{LIMEPY} models are a set of distribution function based equilibrium models that are isothermal
for the most bound stars near the cluster centre and described by polytropes in the outer regions
near the escape energy. The models have been extensively tested against $N$-body models
\citep{Zocchi2016, Peuten2017} and are able to effectively reproduce the effects of mass
segregation. Their suitability for mass modelling globular clusters has been tested on mock data
\citep{Henault-Brunet2019} and they have recently been applied to real datasets as well
\citep[e.g.][]{Gieles2018, Henault-Brunet2020}.


\begin{figure}
	\centering
	\includegraphics[width=0.8\textwidth]{"./figures/limepy_veldisp.png"}
	\includegraphics[width=0.8\textwidth]{"./figures/limepy_numdens.png"}
	\label{fig:1/limepy_models}
	\caption{\ps{TODO: write proper caption} Some fits of limepy models to 47 Tuc}
\end{figure}


The input parameters needed to compute our models include the central concentration parameter $W_0$,
the truncation parameter $g$\footnote{Woolley models \citep{Woolley1954} have $g=0$, King models
	\citep{King1966} $g=1$, and Wilson models \citep{Wilson1975} $g=2$.}, the anisotropy radius $r_a$
which determines the degree of radial anisotropy in the models, $\delta$ which sets the mass
dependence of the velocity scale and thus governs the degree of mass segregation, and finally the
specific mass bins to use as defined by the mean stellar mass ($m_j$) and total mass ($M_j$) of each
bin, which together specify the stellar mass function. In order to scale the model units into
physical units, the total mass of the cluster $M$ and a size scale (the half-mass radius of the
cluster $r_h$) are provided as well.


In their current implementation, these models assume that all objects within the cluster are single
and make no attempt to model the dynamical effects of stellar multiplicity. In this project we adapt
these models to incorporate some of the effects of binary stars under the assumption that all
long-period binaries have been ionized by the present-day. This allows us to treat binary systems as
point-masses and lets us model their dynamics by simply moving some of the mass in stars into
heavier bins according to the specified binary population.



\section{Binary Stars}
\subsection{Binaries in Globular Clusters}

\ps{Discuss binaries in general, mention terminology, then why binaries in clusters are different
	from field binaries, then the dynamical effects of binaries}

In general, the binary systems found within present-day clusters differ significantly from the field
binaries that are more easily observed. In particular, we expect to little no long-period binaries,
on account of them being ionized by the frequent interactions with other cluster members. We
frequently use the terms "hard" and "soft" to describe binaries where "soft binaries" have a binding
energy comparable to the average kinetic energy of a cluster member while "hard binaries" have
larger binding energy. Due to the frequent interactions within clusters we expect that all soft
binaries have long since been ionized by the present-day leaving only a population of hard binaries
with a truncated period distribution compared to field binaries.
\ps{Cite some papers here}

% Binary Burning: \citet{Chatterjee2013}


% Black Hole Burning: \citet{Kremer2019}



The most obvious way that binaries can effect the dynamics of a cluster is through three-body
interactions with other cluster members. When a single star (or another binary) interacts with a
binary system at a close enough range, the binary system will either impart some of its energy to
the ejected star and "harden" or it will capture the approaching star, while ejecting one of its
original components, forming a "harder" binary \citep{Heggie2003}. In either of these cases the
binary system will impart some extra kinetic energy to the ejected star, through this process binary
systems can act as a reserve of kinetic energy for a cluster and are thought to be one of the
primary mechanisms through which core-collapse is halted in some clusters \citep{Chatterjee2013}.
Because the models that we will be focusing on do not model individual objects within the cluster we
will instead focus of the second way that binaries can effect the dynamics of a cluster.

Because binaries are tightly bound, for all interactions except for the very closest, they
effectively act as a single point mass equal to the sum of each component's mass. In this way,
binaries can affect cluster dynamics in much the same way that a large population of heavy remnants
might. Much like black holes and neutron stars, binary systems will migrate to the centre of a
cluster due to the effect of mass-segregation. \citet{Kremer2019} found that a central population of
black holes can fulfill a similar role to binary systems in halting core collapse by injecting
kinetic energy through two-body interactions within the core of the cluster. This same mechanism
could apply with tightly-bound binary systems that have mass-segregated to the centre of the
cluster. This predicted increase in binary fraction as you get closer to the centre of a cluster is
also seen in observations, and is illustrated in Figure \ref{fig:1/radial_binary_fraction} for NGC
3201.

The effect of having a large central population of binaries could be that our models are
overestimating the amount of mass needed in dark mark and therefore overestimating the number of
black holes and high-mass objects in general. Because the gravitational potential in the central
regions is fairly well constrained by kinematic measurements, if we are missing a significant
contribution from binaries, the models may be compensating for this "missing mass" by adding more
mass to the heavy end of the IMF which would lead to overestimation of the number of neutron stars
and black holes.




\begin{figure}
	\centering
	\includegraphics[width=0.8\textwidth]{figures/radial_binarity.pdf}
	\caption{Reproduced from Figure 8 of \citet{Giesers2019} \ps{TODO: Caption, mention what MOCCA is}}
	\label{fig:1/radial_binary_fraction}
\end{figure}

\subsection{Observations of Binary Stars in Globular Clusters}

\paragraph{}
In general, there are two methods used to detect binaries within globular clusters: high-precision
photometric observations and radial velocity surveys.

\paragraph{}
High-precision photometry can be used to detect binaries along the main sequence which have a
significant difference in the mass of their components ( typically these systems have a mass ratio,
$q$, larger than $0.5$). These systems will appear to be raised above the main-sequence when plotted
on a colour-magnitude diagram as their colour will match that of a typical main-sequence star
however their luminosity will be the sum of both components. Figure
\ref{fig:1/main_sequence_binaries} shows the main-sequence of the cluster NGC 2298, the binary stars
in this cluster are visible above the main-sequence according to their mass ratio.
\citet{Milone2012} performed high-precision photometry on several globular clusters using the Hubble
Space Telescope's (HST) Advanced Camera for Surveys and was able to place strong constraints on the
binary fraction for binaries with a mass ratio above $q=0.5$. This method allows for large studies
of binary populations in GCs without the need for dedicated observations but suffers from an
inherent bias towards systems with high mass ratios. Systems with mass ratios below $q=0.5$ are
typically too close to the regular main-sequence to confidently classify as binaries (see Figure
\ref{fig:1/main_sequence_binaries}). This means that studies which employ this method must assume an
underlying mass-ratio distribution if they wish to place any limits on the overall binary fraction
of a cluster.


\begin{figure}
	\centering
	\includegraphics[width=0.8\textwidth]{"./figures/main_sequence_binaries.pdf"}
	\label{fig:1/main_sequence_binaries}
	\caption{\ps{TODO: write proper caption} Reproduced from Figure 1 of \citet{Milone2012}.}
\end{figure}


Large-scale campaigns to measure the radial velocities for many stars in a cluster over many epochs
are another method which can be used to detect binaries in GCs. Systems which are found to have
periodically varying radial velocities can typically be confidently classified as binary systems.
\citet{Giesers2019} used the MUSE integral field spectrograph installed at the European Southern
Observatory's Very Large Telescope to observe several GCs and reported the results for NGC 3201.
Integral field spectrographs provide spatially resolved spectra for the entire field of view of the
detector which enables far more time-efficient surveys than previous methods. Because this method
measures radial velocities over time, periods for the binaries can be accurately determined and
given enough measurements, many other parameters like eccentricity and companion mass can be
accurately constrained in contrast to photometric methods which can only provide the mass ratio.

