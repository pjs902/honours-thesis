


In summary, we have developed a method to include realistic binary populations in \code{LIMEPY}
models. We used this method to fit three sets of models to observations of the globular cluster
47\,Tuc, a set with no binaries, a set with a binary fraction of $2\%$, and a set with a binary
fraction of $10\%$. Despite their different binary fractions, all three sets of models were able to
satisfyingly reproduce all observables and recovered the same structural parameters for the models.
The three sets of models differed primarily in their recovered mass functions and black hole
content. As more binaries are added, few high-mass remnants are required to reproduce the kinematics
of the cluster. This results in a lowered $\alpha_3$ parameter and less mass in black holes.

The implications of this work depend on the binary fraction of the cluster in question. For 47\,Tuc,
our best estimates place the binary fraction at around $2\%$, so this effect is likely negligible. If
however, future observations or models suggest that 47\,Tuc might be host to more binaries, then it
is important that future studies include their effects in their models if they wish to recover
accurate remnant populations.


More generally, for clusters where we expect a high binary fraction like, for example, NGC\,3201,
binaries should certainly be included in any attempts to model the mass distribution of the cluster,
especially if the goal is to constrain the population of heavy remnants within the cluster.



\section{Future Work}



In this work, we only considered binaries made up of two main sequence stars. In reality, binaries
can be formed from any cluster members and binaries where one or both components are heavy remnants
would have an even larger effect on the kinematics of the cluster then main-sequence binaries. The
reason we did not consider this class of binaries in this project is because we have essentially no
constraints on what these populations might look like. The usual photometric methods cannot be used
because there is at most one main-sequence star and radial velocity search will only uncover them if
the binary contains a main-sequence star. It's possible that in this case, we could turn to N-body
or Monte Carlo models to constrain the present-day remnant binary populations, but the binary
populations in these models are likely to be highly dependent on the primordial binary population
and initial conditions of the models.

In the future it would be a nice application of this method to examine a cluster like NGC\,3201
where we know there is likely to be a fairly large binary population. This might allow us to place
stringent and reliable constraints on the black hole population in NGC\,3201 while accounting for
the degeneracy between black hole content and mass in binaries that we have demonstrated can be
important using 47 Tuc as a test case.