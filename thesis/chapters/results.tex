


\section{Results}



\subsection{Previous Results}

As mentioned previously, this project is a continuation of a project done over the previous year in
which we developed a method to use pulsar timing data to constrain our models. The final results of
that project were a set of models that accurately reproduce all observables and fully incorporated
the pulsar data. Figure \ref{fig:nobin_obs_panel} shows the model fits to most of the observables
while Figure \ref{fig:nobin_mass_fun} shows the fit to the stellar mass function data. In both
cases, the models satisfyingly reproduce all observables. The median and $1\sigma$ intervals of the
best fit parameters are listed in Table \ref{tab:parameters_nobin}.

One of the most interesting results of the previous project was the models' ability to constrain the
black hole content within 47\,Tuc. Figure \ref{fig:prev_nobin_BH_dists} shows the distribution of
mass in black hole mass and number of black holes in our set of best fit models, which come out to
$240^{+245}_{-146} \mathrm{M}_\odot$ and $41^{+27}_{-22}$ respectively. Both the total mass and
number are quite well contained especially in comparison to the previous constraints in the
literature \citep[see e.g.][]{Henault-Brunet2020,Weatherford2019}.




\begin{table}
	\centering
	\caption{Best fit parameters with $1\sigma$ intervals.}
	\begin{tabular}{l l}

		\hline
		Parameter                 & Value                  \\
		\hline
		$\Phi_0$                  & $6.62^{+0.11}_{-0.11}$ \\
		$M/10^6 \mathrm{M}_\odot$ & $0.88^{+0.01}_{-0.01}$ \\
		$r_h / pc$                & $6.82^{+0.08}_{-0.07}$ \\
		$\log{r_a / pc}$          & $1.33^{+0.04}_{-0.04}$ \\
		$g$                       & $1.03^{+0.08}_{-0.08}$ \\
		$\delta$                  & $0.37^{+0.02}_{-0.01}$ \\
		$s^2$                     & $0.01^{+0.03}_{-0.01}$ \\
		$F$                       & $3.49^{+0.25}_{-0.22}$ \\
		$\alpha_1$                & $0.47^{+0.06}_{-0.05}$ \\
		$\alpha_2$                & $1.18^{+0.06}_{-0.07}$ \\
		$\alpha_3$                & $2.15^{+0.04}_{-0.04}$ \\
		$BH_{ret} (\%)$           & $0.13^{+0.13}_{-0.08}$ \\
		$d$                       & $4.42^{+0.02}_{-0.02}$ \\
		\hline
	\end{tabular}
	\label{tab:parameters_nobin}
\end{table}

\begin{figure}
	\centering
	\includegraphics[width=0.8\textwidth]{figures/prev_nobin/BH_dists.png}
	\caption{BH distributions}
	\label{fig:prev_nobin_BH_dists}
\end{figure}


\subsection{Low Binary Fraction}
\ps{Describe the results for the low binary fraction model.}


In the models with a $2\%$ binary fraction, we find a similar ability to reproduce all the
observables, Figure \ref{fig:low_bin_model_obs_panel} and Figure \ref{fig:low_bin_model_mass_fun}
show the model fits compared to the data. Once again the models satisfyingly reproduce all
observables.

The black hole content in these models is also quite well contained, though different from the
models without binaries. Figure \ref{fig:low_bin_model_BH_dists} shows the distribution of mass in
black holes and number of black holes \ps{Which no longer gaussian, instead use 95, 99 perentaile
	limit? Why is this so different from $0\%$ case? This makes me want to rerun the fits for no
	binaries with all the updated data and any other changes.}


$114^{+79}_{-144} \mathrm{M}_\odot$
$22^{+13}_{-19}$ Number


The best-fit parameters and their $1\sigma$ intervals are listed in Table \ref{tab:parameters_lowbin}.


\begin{table}
	\centering
	\caption{Best fit parameters with $1\sigma$ intervals.}
	\begin{tabular}{l l}

		\hline
		Parameter                 & Value                  \\
		\hline
		$\Phi_0$                  & $6.28^{+0.10}_{-0.10}$ \\
		$M/10^6 \mathrm{M}_\odot$ & $0.89^{+0.01}_{-0.01}$ \\
		$r_h / pc$                & $6.74^{+0.06}_{-0.06}$ \\
		$\log{r_a / pc}$          & $1.50^{+0.06}_{-0.05}$ \\
		$g$                       & $1.36^{+0.06}_{-0.06}$ \\
		$\delta$                  & $0.43^{+0.02}_{-0.02}$ \\
		$s^2$                     & $0.01^{+0.01}_{-0.00}$ \\
		$F$                       & $3.24^{+0.13}_{-0.12}$ \\
		$\alpha_1$                & $0.37^{+0.02}_{-0.02}$ \\
		$\alpha_2$                & $1.47^{+0.05}_{-0.05}$ \\
		$\alpha_3$                & $2.18^{+0.04}_{-0.04}$ \\
		$BH_{ret} (\%)$           & $0.08^{+0.09}_{-0.05}$ \\
		$d$                       & $4.42^{+0.02}_{-0.02}$ \\
		\hline
	\end{tabular}
	\label{tab:parameters_lowbin}
\end{table}

\begin{figure}
	\centering
	\includegraphics[width=0.9\textwidth]{figures/low_bin_model/obs_panel.png}
	\caption{Observables}
	\label{fig:low_bin_model_obs_panel}
\end{figure}


\begin{figure}
	\centering
	\includegraphics[width=\textwidth]{figures/low_bin_model/mass_fun.png}
	\caption{Mass function}
	\label{fig:low_bin_model_mass_fun}
\end{figure}



\begin{figure}
	\centering
	\includegraphics[width=0.8\textwidth]{figures/low_bin_model/BH_dists.png}
	\caption{BH distributions}
	\label{fig:low_bin_model_BH_dists}
\end{figure}


A binary fraction of $2\%$ results in a total mass in binaries of around $15800 \mathrm{M}_\odot$,
Figure \ref{fig:low_bin_model_Bin_mass} shows the distribution of mass in binaries in our set of
best-fitting models.


\begin{figure}
	\centering
	\includegraphics[width=0.8\textwidth]{figures/low_bin_model/binary_mass.png}
	\caption{Mass in Binaries}
	\label{fig:low_bin_model_Bin_mass}
\end{figure}


\subsection{High Binary Fraction}

As is the case for the models with a low binary fraction, the models with a $10\%$ binary
fraction fit the observables very well. Figures \ref{fig:highbin_obs_panel} and
\ref{fig:highbin_mass_fun} show the model fits compared to the data.

\begin{table}
	\centering
	\caption{Best fit parameters with $1\sigma$ intervals.}
	\begin{tabular}{l l}

		\hline
		Parameter                 & Value                  \\
		\hline
		$\Phi_0$                  & $6.36^{+0.09}_{-0.09}$ \\
		$M/10^6 \mathrm{M}_\odot$ & $0.89^{+0.01}_{-0.01}$ \\
		$r_h / pc$                & $6.77^{+0.06}_{-0.06}$ \\
		$\log{r_a / pc}$          & $1.48^{+0.06}_{-0.05}$ \\
		$g$                       & $1.34^{+0.06}_{-0.06}$ \\
		$\delta$                  & $0.41^{+0.01}_{-0.01}$ \\
		$s^2$                     & $0.01^{+0.01}_{-0.00}$ \\
		$F$                       & $3.16^{+0.13}_{-0.12}$ \\
		$\alpha_1$                & $0.45^{+0.02}_{-0.02}$ \\
		$\alpha_2$                & $1.53^{+0.05}_{-0.04}$ \\
		$\alpha_3$                & $2.46^{+0.05}_{-0.05}$ \\
		$BH_{ret} (\%)$           & $0.17^{+0.18}_{-0.12}$ \\
		$d$                       & $4.43^{+0.02}_{-0.02}$ \\
		\hline
	\end{tabular}
	\label{tab:parameters_highbin}
\end{table}

BH Mass: 80.697 solMass + 80.697 solMass - 120.702 solMass
BH Number: 12.162 + 12.162 - 13.332


\begin{figure}
	\centering
	\includegraphics[width=0.8\textwidth]{figures/high_bin_model/BH_dists.png}
	\caption{BH distributions}
	\label{fig:high_bin_model_BH_dists}
\end{figure}


Now around $81000 \mathrm{M}_\odot$ in binaries.

\begin{figure}
	\centering
	\includegraphics[width=0.8\textwidth]{figures/high_bin_model/binary_mass.png}
	\caption{Binary Mass}
	\label{fig:high_bin_model_Bin_mass}
\end{figure}





\section{Discussion}

\subsection{The effects of the binaries}

Generally mimic a small population of remnants?









\subsection{Conclusion}

\ps{Implications for past/future work}

\ps{Do binaries in df models actually matter?}








\subsection{Future Work}

We're only looking at binaries in main sequence stars. WD binaries are probably a bigger effect, but
we have no data at all to constrain those quantities, so we ignore them. Maybe looking at N-body or
MC models could give some useful constraints.

Would be nice to look at a cluster where we know there is a larger binary population (NGC3201)and
fit with and without binaries to see what the effects are.